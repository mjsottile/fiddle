\begin{abstract}
This paper describes a simple embedded domain specific language,
\FIDDLE, implemented in the host language Haskell for describing 2D
problems based on partial differential equations solved via finite
difference techniques.  \FIDDLE has two goals: facilitate expression
of PDE-based problems that will use a finite-difference technique for their
solution on computers, and allow programmers using the DSL to specify how
they want their program implemeneted at varying layers of abstraction.
We accomplish the first goal by allowing the PDEs and corresponding 
finite difference methods to be expressed in a form as close to their
mathematical specification as possible.  The second goal is accomplished
by a sequence of translations that occur within the DSL in which the high
level mathematical specification is lowered repeatedly to abstract
representations that get closer and closer to the final target language
(in this case C).  The programmer is free to work at any of these
levels if they would prefer to provide expert knowledge beyond that which
is provided by the built in de-sugaring operations that are provided by
\FIDDLE.  Making this levelized control available to the programmer is
critical for allowing performance tuning to take place while still maintaining
high levels of abstraction for the algorithm specification.  We demonstrate
this work in the context of a full 2D implementation of the Navier-Stokes
equations for fluid simulation.
\end{abstract}
