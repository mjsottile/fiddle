\section{Introduction}

Computational physics is largely an activity focused on translating
the equations that describe physical phenomena (such as the
Navier-Stokes equations for fluid flow) into discretized
representations that can be solved on a computer.  The mathematical
language most often used for these systems is that of partial
differential equations (PDEs).  A number of discretization methods
exist to map this continuum definition of the physics in terms of PDEs
into a form that can be solved discretely on a computer.  This paper
describes the design and prototype implementation of an embedded
domain specific language called \FIDDLE that is intended for the
expression of PDE-based problems to be solved via finite-difference
methods.  In this paper we describe the design of \FIDDLE, its
implementation in the host language Haskell, and its application to
the solution of the Navier-Stokes equations for basic incompressible
flow problems.

\subsection{Motivation}

Finite difference, finite volume, and finite elements are the three
most common approaches, all of which share a common origin: the Taylor
series expansion of derivatives that represents them as infinite sums
of difference equations.  Unfortunately, implementation of these PDEs
using one or more of these methods suffers from two major issues.
First, the original PDE or system of PDEs is often lost in the
translation to difference equations based on specific differencing
schemes (e.g., central differences).  The code that implements the
PDEs often bears no relation to the original PDEs other than possible
sharing of variable naming.  Second, the representation of these
difference schemes is most often based on a tedious and error-prone
enumeration of explicit array index references.  This explicit array
indexing makes moving their implementation between concrete
representations very difficult, and limits the freedom of automated
code generation and analysis tools from making changes on behalf of
the user based on knowledge about the high-level system of equations.

The first issue leads to a situation in which the code that implements
a model based on PDEs no longer resembles the original formulation of
the physical problem in the mathematical language used to describe the
physics.  For example, a simple example is the one dimensional
deriviative $\frac{\partial u}{\partial x}$.  This mathematical
formulation, implemented in a discrete solver adopting a finite
difference scheme, may end up ultimately looking like a line of code
such as:

\begin{verbatim}
u[x] = (u[x-1]+u[x])/2;
\end{verbatim}

The concept of the original continuum derivative is lost, and all that
remains is the finite difference scheme represented as a stencil
operator on arrays.  More complex derivatives, such as $\frac{\partial
  uv}{\partial x}$ become much less apparent in their corresponding
implementation as stenciled arrays.  Furthermore, as arrays are finite
structures in programming languages, this code is often surrounded by
code that compensates for boundary conditions - either conditionals or
restricted ranges over which loops are applied to leave certain
elements at the boundary of a region static that contain boundary
condition values.

The second issue that arises is a simple case of the limits of human
programmers.  Consider a three dimensional problem in which one is
computing stencils that involve three indices.  A single sign problem
(typing a {\tt -1} instead of a {\tt +1} in a stencil) can be a very
insidious problem - it is syntactically correct in terms of the
language that implements the stencil, and often the only evidence of
the flaw is incorrect numerical results at the end of the simulation
(or a single time step).

\FIDDLE seeks to address both of these issues, while keeping as an
equal priority the ability of the programmer to perform low level
optimizations to code that practitioners frequently cite as reasons to
remain in low level languages for coding up their production
simulations.  We will describe how \FIDDLE addresses these concerns in
the body of the paper with concrete examples for a 2D Navier-Stokes
problem.

%\subsection{Related work}

%Languages to define programs of the nature described here have existed
%for the entire lifetime of the computing sciences to date.  One can
%argue that the primary purpose (or, at least, use) of Fortran since
%the 1950s was the numerical solution of systems of partial
%differential equations.  Numerous languages have emerged since then
%that address issues of expressivity, parallelism, and performance.
%This work builds on this body of work.  Cites...

% 
