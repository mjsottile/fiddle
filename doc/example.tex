\section{Motivating example}

Throughout this paper we will use examples from the book ``Numerical
Simulation in Fluid Dynamics'' by Griebel, Dornseiffer, and
Neunhoeffer~\cite{griebel98}, and will cite equations from that source
by the numbering scheme used in that book with the prefix GDN to distinguish
them from equation references within this paper.  

In the notation we use here, $u$ and $v$ represent the horizontal and
vertical components of the velocity, $dx$ and $dy$ represent their
spatial discretization in each dimension and temporal discretization
(t), $dt$ represents the temporal discretization, and $p$ represents
the spatially varying pressure (also discretized via $dx$ and $dy$).
We also have a number of parameters, such as the Reynolds number ($Re$),
and a donor-cell scheme weight parameter ($\gamma$).

Let us start with one of the equations that must be solved via finite
difference techniques: the time discretization of the momentum
equation from the Navier-Stokes equations.  We start with the
continuum formulation (GDN-2.2a,b):

\begin{eqnarray}
\frac{\partial u}{\partial t} + \frac{\partial p}{\partial x} & = & \frac{1}{Re} \left(\frac{\partial^2 u}{\partial x^2} + \frac{\partial^2 u}{dy^2}\right) - \frac{\partial(u^2)}{\partial x} - \frac{\partial(uv)}{\partial y} + g_x \nonumber \\
\frac{\partial v}{\partial t} + \frac{\partial p}{\partial y} & = & \frac{1}{Re} \left(\frac{\partial^2 v}{\partial x^2} + \frac{\partial^2 v}{dy^2}\right) - \frac{\partial(uv)}{\partial x} - \frac{\partial v^2}{\partial y} + g_y
\end{eqnarray}

These equations represent a component of the basic physics.  
We will consider the derivatives of the velocity
components $u$ and $v$ first (we will address pressure term $p$ later).  Using Euler's
method, the derivative of the velocity components can be discretized
with respect to time as (GDN-3.28):

\begin{eqnarray}
\label{eq:uv-time-discretization}
\left[ \frac{\partial u}{\partial t} \right]^{(n+1)} & = & \frac{u^{(n+1)} - u^{(n)}}{\delta t} \nonumber \\
\left[ \frac{\partial v}{\partial t} \right]^{(n+1)} & = & \frac{v^{(n+1)} - v^{(n)}}{\delta t}
\end{eqnarray}

A transformation
of Eq.~\ref{eq:uv-time-discretization} leads to a very simple
extension that essentially takes the current value for each state
variable $u$ and $v$, and adds the appropriate derivative scaled by
the time step $\delta t$, with the derivative of the pressure moved to
the right-hand side of the equation via basic algebra (GDN-3.29).

\begin{equation}
\label{eq:timestep_momentum_u}
u^{(n+1)} = u^{(n)} + \delta t \left[ \frac{1}{Re} \left(\frac{\partial^2 u}{\partial x^2} + \frac{\partial^2 u}{dy^2}\right) - \frac{\partial(u^2)}{\partial x} - \frac{\partial(uv)}{\partial y} + g_x - \frac{\partial p}{\partial x} \right]
\end{equation}

\begin{equation}
\label{eq:timestep_momentum_v}
v^{(n+1)} = v^{(n)} + \delta t \left[ \frac{1}{Re} \left(\frac{\partial^2 v}{\partial x^2} + \frac{\partial^2 v}{dy^2}\right) - \frac{\partial(uv)}{\partial x} - \frac{\partial(v^2)}{\partial y} + g_y - \frac{\partial p}{\partial y} \right]
\end{equation}

This leads to a form (GDN-3.36 and GDN-3.37) that represents these
equations without the pressure terms (which are solved elsewhere), by
introducing two additional intermediate variables, $F$ and $G$
(corresponding to the computations for $u$ and $v$ respectively).  We
will write $F$ and $G$ in terms of their position $(i,j)$ in the two
dimensional space that we will solve the equations over.

\begin{eqnarray}
\label{eq:momentum_f}
F_{i,j} & = & u_{i,j} + \delta t \left[ \frac{1}{Re} \left( \left[ \frac{d^2u}{dx^2} \right]_{i,j} + \left[ \frac{d^2u}{dy^2} \right]_{i,j} \right) - \left[ \frac{d(u^2)}{dx} \right]_{i,j} - \left[ \frac{d(uv)}{dy} \right]_{i,j} + g_x \right] \\
\label{eq:momentum_g}
G_{i,j} & = & v_{i,j} + \delta t \left[ \frac{1}{Re} \left( \left[ \frac{d^2v}{dx^2} \right]_{i,j} + \left[ \frac{d^2v}{dy^2} \right]_{i,j} \right) - \left[ \frac{d(uv)}{dx} \right]_{i,j} - \left[ \frac{d(v^2)}{dy} \right]_{i,j} + g_y \right]
\end{eqnarray}

Finally, we can write the equations for the sequence of $u$ and $v$ values 
in terms of $F$ and $G$ along with the pressure term:

\begin{eqnarray}
u^{(n+1)} & = & F - \delta t \frac{\partial p}{\partial x} \nonumber \\
v^{(n+1)} & = & G - \delta t \frac{\partial p}{\partial y}
\end{eqnarray}

At this point, we can begin to explore writing these down in code.  Note that
for the full 2D Navier-Stokes equations, a similar set of equations exist 
that are not described here.  These are present in the \FIDDLE prototype
distribution.

\subsection{Mapping equations to code}

Our goal is to allow the programmer to specify how to solve the system of equations
that describe a physical problem at different levels of abstraction, each of which
is appropriate for the component of the program that they are specifying.  These
include:

\begin{enumerate}
\item The physics based on a composition of differential forms applied to state variables and spatial fields.
\item Local specifications of finite-difference operators representing the differential forms used in the physical specification.
\item Application of equations specified for both the physical model and the corresponding finite difference methods in sequences of code.
\item Abstraction of iterated application of these equations to discretized representations of the space (e.g., regular grids).
\item Explicit separation of the specification of local differencing methods from their concrete representation on a computer (e.g., dense 2D array).
\end{enumerate}

Before we continue, some terminology should be clarified.  Consider the
definition for a 1D finite difference equation:

\begin{equation}
\left[\frac{du}{dx}\right]_i = \frac{u_i - u_{i-1}}{dx}
\end{equation}

We say that this is specified in a \emph{local form} because the specification is
valid for any index $i$.  The references to neighboring portions of the space are
relative to the \emph{central index} $i$.  This is important as it removes
any reference to a size of the space and any assumptions about the way that
space is represented.  It simply states that given an ordering of points (such that
indices and their neighbors are well defined), this definition of the approximation
for the derivative can be applied.

In \FIDDLE, we adopt a local view for equation specification.  Instead of writing
equations in terms of some indices $i, j, k$ and so on, we write them in terms of
offsets from a central index.  So for example, $(i-1,j+1)$ becomes $(-1,1)$, 
and $(i,j)$ becomes $(0,0)$.  \FIDDLE provides a special operator, {\tt (@@)}, that
binds local index offsets to variables that represent spatial fields.  As such,
$u_{i-1,j}$ is written {\tt u@@(-1,0)}.  The binding of these offsets to concrete
indices is performed as part of the translation process from the high-level DSL to
the lower-level intermediate representation and subsequent implementation.

Let's start with the first goal: the specification of the physical model in terms
of abstract differential forms applied to state variables and spatial fields.
Given the equations in Eq.~\ref{eq:momentum_f} and Eq.~\ref{eq:momentum_g}, we can write their right-hand sides in code\footnote{The function names, such as {\tt eq336} are based on the equation numbering scheme from the GDN text.} as:

\begin{lstlisting}
eq336 u v re dx dy dt gx gamma = 
  u@@(0,0) +
  (dt * ((1 / re) * ((d2udx2 u dx) + (d2udy2 u dy)) -
         (du2dx u dx gamma) - (duvdy u v dy gamma) + gx))

eq337 u v re dx dy dt gy gamma =
  v@@(0,0) +
  (dt * ((1 / re) * ((d2vdx2 v dx) + (d2vdy2 v dy)) -
         (duvdx u v dx gamma) - (dv2dy v dy gamma) + gy))
\end{lstlisting}

Now, these components of the momentum equation are written in terms of derivatives.
As can be seen in the code for {\tt eq336}, there is a direct correspondence between
the derivative in the mathematical formula and the code.  This then leads to our
second goal, the local specification of finite-difference operators that correspond to
the differential forms used to encode the physical model.  For example, a simple 
first derivative in the horizontal direction in a 2D space is written:

\begin{equation}
\left[\frac{du}{dx}\right]_{i,j} = \frac{u_{i,j} - u_{i-1,j}}{dx}
\end{equation}

This corresponds to the following code:

\begin{lstlisting}
dudx u dx = (u@@(0,0) - u@@(-1,0)) / dx
\end{lstlisting}

More interesting examples come about when one encodes, say, a central difference
scheme combined with a donor cell scheme (please refer to the GDN text for
details on how this is derived).  Consider the following finite difference
equation that corresponds to one of the derivatives appearing in the
momentum equation:

\begin{eqnarray}
\left[\frac{d(uv)}{dx}\right]_{i,j} & = & \frac{1}{dy} \left( \frac{v_{i,j} + v_{i+1,j}}{2} \frac{u_{i,j} + u_{i,j+1}}{2} - \frac{|v_{i,j-1} + v_{i+1,j-1}|}{2}\frac{u_{i,j-1} + u_{i,j}}{2}\right) + \nonumber \\
& & \gamma \frac{1}{dy} \left( \frac{|v_{i,j} + v_{i+1,j}|}{2} \frac{u_{i,j} - u_{i,j+1}}{2} - \frac{|v_{i,j-1} + v_{i+1,j-1}|}{2}\frac{u_{i,j-1} - u_{i,j}}{2}\right)
\end{eqnarray}

In code:

\begin{lstlisting}
duvdy u v dy gamma = 
  (1 / dy) * (((v@@(0,0) + v@@(1,0)) / 2) * ((u@@(0,0) + u@@(0,1)) / 2) -
              ((v@@(0,-1) + v@@(1,-1)) / 2) * ((u@@(0,-1) + u@@(0,0)) / 2))
  + (gamma / dy) *
    ( ((abs (v@@(0,0) + v@@(1,0))) / 2) * ((u@@(0,0) - u@@(0,1)) / 2) -
      ((abs (v@@(0,-1) + v@@(1,-1))) / 2) * ((u@@(0,-1) - u@@(0,0)) / 2))
\end{lstlisting}

\FIDDLE allows the code specification to have a direct correspondence to the
mathematical specification.  An interesting consequence of this is that as part
of our early testing of the DSL, we wrote a pretty-printer that emitted
\LaTeX{ } code instead of program source code from these equational specifications.
The result was equivalent to the original mathematical specification, differing
only in the use of parenthesis in the computer-generated \LaTeX.  This simple
test led us to the interesting conclusion that an added benefit of this approach
for program specification is that as part of the documentation generation
process, the mathematics that are actually implemented in the program can
be emitted in a form that is human readable to aid in verifying that the program
matches the physical model specification.

Now, given a set of difference operators and equational definitions, we want to
compose them together as functions that operate over the data (our third goal).
The intent here is to bring the high level mathematical specification together with
a high-level specification of the computational structure of the code.  We still
avoid committing to low level implementation details (such as parallelism strategies
or specific loop constructs), but we do commit to a sequence of abstract operations.

\begin{lstlisting}
comp_FG f g u v re dx dy dt gx gy gamma =
  Function
    (TYPrim TYVoid)
    "comp_fg"
    [arrName f, arrName g, arrName u, arrName v, re, dx, dy, dt, gx, gy, gamma]
    (ScopedRegion Nothing
       [nosyms theloop])
  where
    rbounds = (1, ArrayBound f UpperRows, Nothing)
    cbounds = (1, ArrayBound f UpperCols, Nothing)

    theloop = ForRange2 rbounds cbounds body

    body = [\i j -> (Indexed2D f i j) ===
                    (eq336 (Indexable2D u i j) (Indexable2D v i j)
                           vre vdx vdy vdt vgx vgamma),
            \i j -> (Indexed2D g i j) ===
                    (eq337 (Indexable2D u i j) (Indexable2D v i j)
                           vre vdx vdy vdt vgy vgamma)]

    [vre, vdx, vdy, vdt, vgx, vgy, vgamma] = map var [re,dx,dy,dt,gx,gy,gamma]
\end{lstlisting}

Within this code, we see the fourth goal accomplished: we abstract the manner
by which iteration occurs via the operator {\tt ForRange2} which, given two
index ranges, applies the equations bound to concrete variables over the
index ranges.  The manner by which this iterator is implemented is left to the
phase in which this specification is transformed to a lower level representation.
We will explain some of the constructs that have not been described so far in the 
next section.  It is also important to note that we have also addressed the fifth
goal of avoiding committing to specific representations yet.  We are writing
the code only in terms of ``Indexable'' variables and ``Indexed'' representations.

To complete the discussion, we can look at the concrete code that this is translated to
by the DSL translator.  The code contains the fully expanded physical model equations based on the finite difference operators.  The indexing scheme is mapped to a concrete
set of indices and nested for-loops.  The array representation is also made concrete
at translation time, in this case to a structure-based representation modeled on
Fortran array descriptors.  One consequence of the design of \FIDDLE is that the
representation of the arrays and array iterators can be changed \emph{without
modification to the physics and finite difference operator specification}.  This
demonstrates the intended separation of high-level physical and mathematical
specification from specific implementation choices.

\begin{lstlisting}
void 
comp_fg(FloatArray2D_t * f, FloatArray2D_t * g, FloatArray2D_t * u, 
        FloatArray2D_t * v, float re, float dx, float dy, float dt, 
        float gx, float gy, float gamma)
{
  for (int i__ = 1; i__ <= f->dimupper[0]; i__++) {
    for (int j__ = 1; j__ <= f->dimupper[1]; j__++) {
      f->data[i__][j__] = (u->data[i__][j__]) + ((dt) * ((((((1) / (re)) * (((((u->data[(i__) + (1)][j__]) - ((2) * (u->data[i__][j__]))) + (u->data[(i__) - (1)][j__])) / ((dx) * (dx))) + ((((u->data[i__][(j__) + (1)]) - ((2) * (u->data[i__][j__]))) + (u->data[i__][(j__) - (1)])) / ((dy) * (dy))))) - ((((1) / (dx)) * (((((u->data[i__][j__]) + (u->data[(i__) + (1)][j__])) / (2)) * (((u->data[i__][j__]) + (u->data[(i__) + (1)][j__])) / (2))) - ((((u->data[(i__) - (1)][j__]) + (u->data[i__][j__])) / (2)) * (((u->data[(i__) - (1)][j__]) + (u->data[i__][j__])) / (2))))) + (((gamma) / (dx)) * ((((fabs((u->data[i__][j__]) + (u->data[(i__) + (1)][j__]))) / (2)) * (((u->data[i__][j__]) - (u->data[(i__) + (1)][j__])) / (2))) - (((fabs((u->data[(i__) - (1)][j__]) + (u->data[i__][j__]))) / (2)) * (((u->data[(i__) - (1)][j__]) - (u->data[i__][j__])) / (2))))))) - ((((1) / (dy)) * (((((v->data[i__][j__]) + (v->data[(i__) + (1)][j__])) / (2)) * (((u->data[i__][j__]) + (u->data[i__][(j__) + (1)])) / (2))) - ((((v->data[i__][(j__) - (1)]) + (v->data[(i__) + (1)][(j__) - (1)])) / (2)) * (((u->data[i__][(j__) - (1)]) + (u->data[i__][j__])) / (2))))) + (((gamma) / (dy)) * ((((fabs((v->data[i__][j__]) + (v->data[(i__) + (1)][j__]))) / (2)) * (((u->data[i__][j__]) - (u->data[i__][(j__) + (1)])) / (2))) - (((fabs((v->data[i__][(j__) - (1)]) + (v->data[(i__) + (1)][(j__) - (1)]))) / (2)) * (((u->data[i__][(j__) - (1)]) - (u->data[i__][j__])) / (2))))))) + (gx)));
                        ;
      g->data[i__][j__] = (v->data[i__][j__]) + ((dt) * ((((((1) / (re)) * (((((v->data[(i__) + (1)][j__]) - ((2) * (v->data[i__][j__]))) + (v->data[(i__) - (1)][j__])) / ((dx) * (dx))) + ((((v->data[i__][(j__) + (1)]) - ((2) * (v->data[i__][j__]))) + (v->data[i__][(j__) - (1)])) / ((dy) * (dy))))) - ((((1) / (dx)) * (((((u->data[i__][j__]) + (u->data[i__][(j__) + (1)])) / (2)) * (((v->data[i__][j__]) + (v->data[(i__) + (1)][j__])) / (2))) - ((((u->data[(i__) - (1)][j__]) + (u->data[(i__) - (1)][(j__) + (1)])) / (2)) * (((v->data[(i__) - (1)][j__]) + (v->data[i__][j__])) / (2))))) + (((gamma) / (dx)) * ((((fabs((u->data[i__][j__]) + (u->data[i__][(j__) + (1)]))) / (2)) * (((v->data[i__][j__]) - (v->data[(i__) + (1)][j__])) / (2))) - (((fabs((u->data[(i__) - (1)][j__]) + (u->data[(i__) - (1)][(j__) + (1)]))) / (2)) * (((v->data[(i__) - (1)][j__]) - (v->data[i__][j__])) / (2))))))) - ((((1) / (dy)) * (((((v->data[i__][j__]) + (v->data[i__][(j__) + (1)])) / (2)) * (((v->data[i__][j__]) + (v->data[i__][(j__) + (1)])) / (2))) - ((((v->data[i__][(j__) - (1)]) + (v->data[i__][j__])) / (2)) * (((v->data[i__][(j__) - (1)]) + (v->data[i__][j__])) / (2))))) + (((gamma) / (dy)) * ((((fabs((v->data[i__][j__]) + (v->data[i__][(j__) + (1)]))) / (2)) * (((v->data[i__][j__]) - (v->data[i__][(j__) + (1)])) / (2))) - (((fabs((v->data[i__][(j__) - (1)]) + (v->data[i__][j__]))) / (2)) * (((v->data[i__][(j__) - (1)]) - (v->data[i__][j__])) / (2))))))) + (gy)));
                        ;
    }
  }
        ;
}
\end{lstlisting}
